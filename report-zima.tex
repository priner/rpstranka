\documentclass[a4paper, 11pt]{article}
\usepackage{csquotes}
\usepackage[slovak]{babel}

\title{
    Ročníkový projekt \\
    \large Report za zimný semester
}
\author{Ján Priner}
\date{}

\begin{document}

\maketitle

\section{Čo som spravil ja}

Vytvoril som triedy solverov (pre helpmate a selfmate), a v nich implementoval
rekurzívne prehľadávanie stavov hry. Triedu SolverFactory ktorá z FENu vytvorí
inštanciu solvera (časť parsovania FENu implementovaná už v engine).
Načítavanie vstupu a vypisovanie výstupu.

\section{Limitácie/priestor na zlepšenie}

Program v súčasnosti zvláda riešiť problémy \enquote{selfmate in n moves}
a \enquote{helpmate in n moves} iba pre veľmi malé $n$ ($\leq 3$). Je to hlavne
preto, že prehľadávaný priestor je nesmierne veľký. V jednom ťahu si hráč
zvyčajne vyberá z okolo 30 ťahov, to pre $n=3$ je okolo $10^9$ stavov.

Niektoré veci by sa dali lepšie zoptimalizovať. Tiež by sme v budúcnosti mohli
na zrýchlenie prehľadávania využiť multithreading.

\end{document}
