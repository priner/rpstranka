\documentclass[a4paper, 11pt]{article}
\usepackage{csquotes}
\usepackage[slovak]{babel}

\title{
    Ročníkový projekt \\
    \large Report za letný semester
}
\author{Ján Priner}
\date{}

\begin{document}

\maketitle

Vymenil som SAT solver knižnicu lingeling za knižnicu glucose. Tú je možné
skompilovať aj pre Windows. Bolo treba vykonať ešte niekoľko drobných zmien
aby celý program fungoval na systéme Windows (cez cygwin).

Sprehľadnil som vypisovanie výsledkov:
\begin{itemize}
\item
Farby sú teraz pomenované tak, že vrcholy tetrahédra sú pomenované 1, 2, 3, 4
a hrany podľa koncových vrcholov 12, 13, 14, 23, 24, 34 namiesto pôvodného
neprehľadného označenia, kde farby boli iba očíslované od 0 po 9.
\item
Pri vypisovaní prechodov multipólu Štvorice farieb ktoré sú ekvivalentné
už nevypisujeme dvakrát (ekvivalentné sú ak sa dajú prečíslovať vrcholy
tetraéderu tak aby boli rovnaké)
\end{itemize}

Vytvoril som tiež nástroj, ktorý skúma snarky z ktorých bol odstránený
5-cyklus (teda mu zostalo 5 visiacich hrán). Program skúma to, že keď ten
zvyšok grafu zafarbíme tetraedrálnou Steinerovskou konfiguráciou, aké
pätice sú prípustné na visiacich hranách. Ako aj ostatné nástroje aj tento
úlohu rieši pomocou SAT solvera.

Päticu ktoré vieme dostať z inej prečíslovaním
vrcholov tetrahédra, alebo cyklickou rotáciou pätice, alebo prevrátením pätice
považujeme za totožné a vypisujeme iba raz. Naviac o každej pätici program
vypíše či je pentagonálna, pentagramálna, alebo adverzná.

\end{document}
